\documentclass{beamer}
\usetheme{Madrid}

\title{High-Dimensional Option Pricing with PINNs}
\subtitle{Solving the Multi-Asset Black-Scholes Equation}
\author{Yon Ploj}
\institute{DiffProg project}
\date{March 2025}

\begin{document}

\frame{\titlepage}

\begin{frame}{Options}
\begin{itemize}
    \item An \textbf{option} is a financial contract that gives the holder the right (but not the obligation) to buy or sell an asset at a specified price before a certain date.
    \item \textbf{Call option}: right to buy; \textbf{Put option}: right to sell.
    \item Used by financial institutions to hedge investments.
    \item \textbf{Multi-asset options} depend on multiple underlying assets.
    \item Pricing options requires solving mathematical models like the \textbf{Black-Scholes equation}.
\end{itemize}
\end{frame}

\begin{frame}{The Black-Scholes equation for multiple assets}
\small
\[
\frac{\partial u}{\partial t}
+ \sum_{i=1}^{d} r x_i \frac{\partial u}{\partial x_i}
+ \frac{1}{2} \sum_{i=1}^{d} \sum_{j=1}^{d} \rho_{ij} \sigma_i \sigma_j x_i x_j \frac{\partial^2 u}{\partial x_i \partial x_j}
- r u = 0
\]

\begin{itemize}
    \item $d$: number of assets (dimensions)
    \item $\sigma_i$: volatility of asset $i$
    \item $\rho_{ij}$: correlation coefficient between assets $x_i$ and $x_j$
    \item $r$: risk-free interest rate
    \item Terminal condition: payoff at maturity
\end{itemize}
More complicated than the regular Black-Scholes equation, but it takes into account asset \textbf{interdependencies}.
\end{frame}

\begin{frame}{Motivation}
\begin{itemize}
    \item Multi-asset options depend on several underlying assets (e.g., basket options).
    \item Pricing them involves solving high-dimensional PDEs (one dimension per asset).
    \item Traditional methods fail in high dimensions due to the \textbf{curse of dimensionality}.
    \item We use PINNs with Stochastic Dimension Gradient Descent (SDGD) to overcome this.
\end{itemize}
\end{frame}

\begin{frame}{SDGD + PINN}
\begin{itemize}
    \item \textbf{Stochastic Dimension Gradient Descent (SDGD):}
    \begin{itemize}
        \item Randomly sample $m$ (out of $d$) dimensions each iteration.
        \item Estimate PDE loss using partial derivatives from sampled dims.
        \item Reduces per-iteration cost from $O(d)$ to $O(m)$.
    \end{itemize}
    \item PINNs compute gradients via autograd; SDGD avoids full-dimension derivatives every step.
\end{itemize}
\end{frame}

\begin{frame}{Goals}
\begin{itemize}
    \item Solve Black-Scholes PDE.
    \item Measure the speed and accuracy benefits of SDGD.
    \item Show PINNs can tackle real financial PDEs previously considered intractable.
\end{itemize}
\end{frame}

\begin{frame}{Blank slide}
\centering

\end{frame}

\end{document}